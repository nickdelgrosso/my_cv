\documentclass{article}%
\usepackage[T1]{fontenc}%
\usepackage[utf8]{inputenc}%
\usepackage{lmodern}%
\usepackage{textcomp}%
\usepackage{lastpage}%
\usepackage{marginnote}%
\reversemarginpar%
\usepackage{graphicx}%
\usepackage[nochapters]{classicthesis}%
\usepackage[LabelsAligned]{currvita}%
\usepackage{hyperref}%
\hypersetup{colorlinks, breaklinks, urlcolor=Maroon, linkcolor=Maroon}%
\newlength{\datebox}%
\settowidth{\datebox}{Tuebingen, Germany}%
\renewcommand{\cvheadingfont}{\LARGE\color{Maroon}}%
\newcommand{\SubHeading}[1]{\vspace{1em}\noindent\spacedlowsmallcaps{#1}\vspace{0.7em}\\}%
\newcommand{\Email}[1]{\href{mailto:#1}{#1}}%
\newcommand{\MarginText}[1]{\marginpar{\raggedleft\small#1}}%
\newcommand{\Description}[1]{\hangindent=2em\hangafter=0\footnotesize{#1}\par\normalsize\vspace{1em}}%
\newcommand{\DescMarg}[2]{\noindent\hangindent=2em\hangafter=0 \parbox{\datebox}{\small} \MarginText{#1} #2 \vspace{0.3em}\\}%
\newcommand{\HeaderOnly}[2]{\noindent\hangindent=2em\hangafter=0 \parbox{\datebox}{\small \textit{#1}}\hspace{1.5em} #2 \vspace{0.5em}\\}%
\newcommand{\EntryHeader}[3]{\noindent\hangindent=2em\hangafter=0 \parbox{\datebox}{\small \textit{#2}}\hspace{1.5em} \MarginText{#1} #3 \vspace{0.5em}}%
\newcommand{\NewEntry}[4]{\EntryHeader{#1}{#2}{#3}\\\Description{#4}}%
%
%
%
\begin{document}%
\normalsize%
\thispagestyle{empty}%
\raggedright%
\begin{cv}{\spacedallcaps{Nicholas A. Del Grosso}}%
\vspace{2em}%
\SubHeading{Personliche Daten}%
\HeaderOnly{Adresse}{Karl{-}Witthalm{-}Str. 3, 81375 München}%
\HeaderOnly{Telefone}{+49 170 8253289}%
\HeaderOnly{Email}{\Email{delgrosso.nick@gmail.com}}%
\SubHeading{Ziele}%
\begin{itemize}%
\item%
Andere dazu inspirieren, durch Mentoring, Unterricht und Führung großartige Dinge zu erreichen.%
\end{itemize}%
\begin{itemize}%
\item%
Technische Fähigkeiten in den verschiedensten Bereichen zu entwickeln, um qualitativ hochwertige Forschung an Instituten mit begrenzten Ressourcen durchzuführen.%
\end{itemize}%
\begin{itemize}%
\item%
Unterstützung offener Wissenschaft durch den Aufbau von Werkzeugen und das Unterrichten von Forschungsmethoden, die reproduzierbare Forschung fördern.%
\end{itemize}%
\begin{itemize}%
\item%
Erhalten Sie Lehr{-}, Projektmanagement{-} und Laborerfahrung, um eines Tages ein kompetenter Universitätsprofessor zu werden.%
\end{itemize}%
\SubHeading{Ausbildung}%
\NewEntry{Oct 2014 {-} Present}{PhD. Neurowissenschaft}{Graduate School of Systemic Neurosciences, Lüdwig{-}Maximillians Üniversität}{}%
\NewEntry{Aug 2012}{M.Sc. Neurowissenschaft}{Max Planck International Research School, Graduate School of Neural and Behavioural Sciences}{}%
\NewEntry{May 2010}{B.Sc. Psychologie}{Wittenberg University}{}%
\SubHeading{Forschungserfahrung}%
\NewEntry{May 2013 {-}\newline%
  Present}{Ludwig{-}Maximillians Universität}{Prof. Dr. Anton Sirota}{Programmierte eine 3D{-}Grafik{-}Engine in Python, um ein Virtual{-}Reality{-}System für frei bewegliche Ratten aufzubauen, konzipierte und realisierte kognitionswissenschaftliche Experimente und testete die Verallgemeinerbarkeit der Virtual{-}Reality{-}Forschung an ihren realen Gegenstücken; betreute sechs Studenten in Programmier{-}, Ingenieur{-} und kognitionswissenschaftlichen Projekten organisierte wöchentliche Journalclubs, organisierte geplante soziale Veranstaltungen und Retreats und bestellte neue Laborgeräte.}%
\SubHeading{Berufserfahrung}%
\NewEntry{}{Freiberuflich}{Wissenschaftlicher Berater und Trainer}{Trainiere ich Forscher in Programmierung, Experimentierdesign und wissenschaftlichen Schreibfähigkeiten, indem Sie sie dabei unterstützen, ihre eigenen Lösungen für Forschungsprobleme zu entwickeln und einwöchige Schulungen für ihre Forschungsinstitute durchzuführen.}%
\SubHeading{wissenschaftliche Publikationen}%
\Description{Nicholas A. Del Grosso, Justin J. Graboski, Weiwei Chen, Eduardo Blanco Hernández, Anton Sirota. ``Virtual Reality system for freely{-}moving rodents.'' bioRxiv 161232. July 2017; doi=https://doi.org/10.1101/161232}%
\Description{Broetz D., Del Grosso, N.A., Rea M., Ramos{-}Murguialday, A., Soekadar S.R., Birbaumer, N. ``A New Hand Assessment Instrument for Severely Affected Stroke Patients.''  Journal of Neurorehabilitation. 2014; 34(3), 409{-}27.}%
\Description{Benoit, J.B., Del Grosso, N.A., Yoder, J.A., Denlinger, D.L. ``Resistance to Dehydration between Bouts of Blood Feeding in the Bed Bug, Cimex Lectularius, is Enhanced by Water Conservation, Aggregation, and Quiescence.'' American Journal of Tropical Medical Hygience. May 2007; 76(5), 987{-}93.}%
\SubHeading{Konferenzpublikationen}%
\NewEntry{September 2018}{Harvard{-}LMU Young Scientists Forum}{Testing CAVE virtual reality systems for use in animal behavior research}{}%
\NewEntry{November 2017}{Society for Neuroscience}{Generalized Rat Spontaneous Behavior in a CAVE Experimental Setup.}{}%
\NewEntry{July 2017}{PyData Barcelona}{The Neuroscience Lab; A Tour Through the Eyes of a Pythonista}{}%
\NewEntry{November 2016}{Munich Interact}{Tracking Rats Exploring a Virtual World; Do They Believe what they See?}{}%
\NewEntry{July 2016}{FENS Forum of Neuroscience}{Probing Rodent Perception of Virtual Environments with Freely{-}Moving Virtual Reality}{}%
\NewEntry{June 2015}{Synergy Munich}{ratCAVE, A Novel Virtual Reality System for Freely{-}Moving Rodents}{}%
\NewEntry{March 2015}{Interact Munich}{Demonstrating a Freely{-}Moving Virtual Reality Approach for Rodent Research}{}%
\NewEntry{Nov 2014}{Society for Neuroscience}{ratCAVE, A Novel Virtual Reality System for Freely{-}Moving Rodents.}{}%
\NewEntry{Nov. 2012}{NENA Tübingen}{Interpreting (M)EEG, A First Look at Dynamic Causal Modeling.}{Introduced a probabilistic nonlinear modeling framework for interpretation of MEG and EEG data, along with the results of a pilot study in which we applied the approach.}%
\NewEntry{Nov. 2011}{NENA Tübingen}{The Intrinsic Bias During the Blind{-}Walking Task is Not Caused by an Aberrant Intrinsic Ground{-}Slope Model.}{}%
\SubHeading{Sonstiges}%
\begin{itemize}%
\item%
\textbf{Sprachkenntnisse}: English (Mother Tongue), German (Level B1), French (Level A1-2)%
\end{itemize}%
\begin{itemize}%
\item%
\textbf{Programmieren}: Python, Matlab, C-Sharp, GLSL, R, LabView, C, Bash/Linux, LaTeX%
\end{itemize}%
\begin{itemize}%
\item%
\textbf{Grafik}: Psychopy, Neurobs Presentation, Psychophysics Toolbox, OpenGL, Pyglet, SuperLab, RatCAVE, Blender3D, Adobe Suite (Photoshop, Illustrator, and InDesign), Google SketchUp, GIMP, Inkspace%
\end{itemize}%
\begin{itemize}%
\item%
\textbf{Statistik}: Statistical Parametric Mapping (SPM), SPSS, R, Matlab Statistics Toolbox, Fieldtrip, gTec Analyze, BrainVision Analyzer%
\end{itemize}%
\begin{itemize}%
\item%
\textbf{Labor Fähigkeiten}: Ratte Neurochirurgie, Tierisches Verhaltenstraining (Ratten und Affen), In-vivo-Elektrophysiologie (Einnadelelektroden, chronisch implantierte Elektrodenarrays, nichtinvasive Arrays von EEG-Elektroden und MEG-Sensoren), Basiselektronik, Bequem mit benutzerdefinierten Laboreinrichtungen%
\end{itemize}%
\begin{itemize}%
\item%
\textbf{EEG System Erhahrungen}: BrainProducts, g.tec, Grass Instruments, CTF%
\end{itemize}%
\SubHeading{Auszeichnungen}%
\DescMarg{Oktober 2017}{Hackathon 3. Platz Gewinner und "Most Creative Team" Award beim Burda Bootcamp Event "Gesundheit und Fitness Hackathon"}%
\DescMarg{July 2017}{Hackathon Track Gewinner beim Media Lab Bayern Event "FutureLab {-} Smart Home trifft Journalismus"}%
\DescMarg{April 2017}{Hackathon Gewinner beim Burda Bootcamp Event "Love Hackathon"}%
\DescMarg{2016}{Best Talk Award auf der Interact München Conference}%
\DescMarg{2015}{Best Poster Award auf der Interact Munich Conference}%
\DescMarg{2011}{National Science Foundation Graduate Research Graduiertenstipendium}%
\DescMarg{2008}{NSF Neuroscience REU Fellowship at Duke Graduiertenstipendium}%
Full List of Positions and Publications Available Upon Request.%
\includegraphics[width=5cm]{images/Signaturetransparant.png}%
\end{cv}%
\end{document}