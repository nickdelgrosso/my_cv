\documentclass{article}%
\usepackage[T1]{fontenc}%
\usepackage[utf8]{inputenc}%
\usepackage{lmodern}%
\usepackage{textcomp}%
\usepackage{lastpage}%
\usepackage{marginnote}%
\reversemarginpar%
\usepackage{graphicx}%
\usepackage[nochapters]{classicthesis}%
\usepackage[LabelsAligned]{currvita}%
\usepackage{hyperref}%
\hypersetup{colorlinks, breaklinks, urlcolor=Maroon, linkcolor=Maroon}%
\newlength{\datebox}%
\settowidth{\datebox}{Tuebingen, Germany}%
\renewcommand{\cvheadingfont}{\LARGE\color{Maroon}}%
\newcommand{\SubHeading}[1]{\vspace{1em}\noindent\spacedlowsmallcaps{#1}\vspace{0.7em}\\}%
\newcommand{\Email}[1]{\href{mailto:#1}{#1}}%
\newcommand{\MarginText}[1]{\marginpar{\raggedleft\small#1}}%
\newcommand{\Description}[1]{\hangindent=2em\hangafter=0\footnotesize{#1}\par\normalsize\vspace{1em}}%
\newcommand{\DescMarg}[2]{\noindent\hangindent=2em\hangafter=0 \parbox{\datebox}{\small} \MarginText{#1} #2 \vspace{0.3em}\\}%
\newcommand{\HeaderOnly}[2]{\noindent\hangindent=2em\hangafter=0 \parbox{\datebox}{\small \textit{#1}}\hspace{1.5em} #2 \vspace{0.5em}\\}%
\newcommand{\EntryHeader}[3]{\noindent\hangindent=2em\hangafter=0 \parbox{\datebox}{\small \textit{#2}}\hspace{1.5em} \MarginText{#1} #3 \vspace{0.5em}}%
\newcommand{\NewEntry}[4]{\EntryHeader{#1}{#2}{#3}\\\Description{#4}}%
%
%
%
\begin{document}%
\normalsize%
\thispagestyle{empty}%
\raggedright%
\begin{cv}{\spacedallcaps{Nicholas A. Del Grosso}}%
\vspace{2em}%
\SubHeading{Personliche Daten}%
\HeaderOnly{Adresse}{Karl{-}Witthalm{-}Str. 3, 81375 München}%
\HeaderOnly{Telefone}{+49 170 8253289}%
\HeaderOnly{Email}{\Email{delgrosso.nick@gmail.com}}%
\SubHeading{Ausbildung}%
\NewEntry{Oct 2014 {-} Present}{PhD. Neurowissenschaft}{Graduate School of Systemic Neurosciences, Lüdwig{-}Maximillians Üniversität}{}%
\NewEntry{Aug 2012}{M.Sc. Neurowissenschaft}{Max Planck International Research School, Graduate School of Neural and Behavioural Sciences}{}%
\NewEntry{May 2010}{B.Sc. Psychologie}{Wittenberg University}{}%
\SubHeading{Lehrerfahrung}%
\NewEntry{July 2018}{Programming Instructor}{Lehrstatistik mit R}{Dieser viertägige Workshop ist ein intensiver R{-}Kurs, der den Professoren der Psychologieabteilung der Kwantlen Universität in Vancouver, Kanada, erteilt wird. In diesem Kurs lernten sie die R{-}Programmiersprache und lernten, wie sie damit statistisches Unterrichtsmaterial erstellen können.}%
\NewEntry{April 2018}{Veranstalter}{Munich Science Slam}{Organisation eines Science{-}Slam Verananstaltung in einer lokalen Veranstaltungshalle mit 14 Referenten aus 5 Instituten, um Vorträge zu halten. Integriert ein Echtzeit{-}Bewertungssystem in die Veranstaltung für die 60 Zuschauer für Feedback verwenden. Diese Veranstaltung war erfolgreich und wurde im Oktober 2018 wiederholt}%
\NewEntry{April 2018 {-} Present}{Soft Skills Trainer}{Präsentationsfähigkeiten für Wissenschaftler}{Mit diesem 1{-}2{-}tägigen Workshop unterrichtete ich effektive Sprechorganisation und skizzierte Fähigkeiten für wissenschaftliche Präsentationen. Wissenschaftler gewinnen Vertrauen, indem sie lernen, sich auf ihre Ziele zu konzentrieren und durch wiederholtes Üben überzeugend zu sprechen.}%
\NewEntry{October 2017}{3D Graphics Instructor}{Tierverfolgung und VR Bootcamp}{Ich habe einen internationalen, einwöchigen Workshop über die Kombination von Tierverfolgung mit Methoden der Bildverarbeitung und 3D{-}Grafikanwendungen gelehrt, um virtuelle Realitätssysteme für frei bewegliche Tiere zu bauen. Neben theoretischen Vorlesungen über Mathematik und Technik hinter Virtual Reality Systemen schrieb ich Tutorials für Software, die ich geschrieben habe, um die Konzepte zu unterrichten, aus denen die Teilnehmer, bestehend aus Doktoranden, Postdocs und Professoren, ihre eigenen Prototyp{-}VR{-}Systeme für Ameisen entwickelten.}%
\NewEntry{Fall 2017}{Veranstalter}{PyData Munich}{Ich habe ein lokales Kapitel für die globale PyData{-}Organisation wiederbelebt und mich mit Technologieunternehmen in München (z. B. Google, Nokia, TNG Consulting, JetBrains und Wayra) zusammengetan, um mithilfe der MeetUp{-}Plattform eine datenwissenschaftliche Lehrgemeinschaft aufzubauen. Diese Unternehmen veranstalten jetzt zweiwöchentliche Tutorien in ihren Veranstaltungsräumen, sponsern jede Veranstaltung und bieten Raum für Universitätsforscher und Experten der Technologiebranche, um sich zu treffen, zu interagieren und gemeinsam zu lernen.}%
\NewEntry{Summer 2017}{Veranstalter}{Super Python Talks for Life Science}{Ich organisierte eine zweiwöchige Seminarreihe für die Vermittlung von Datenanalysen mittlerer Ebene und Python{-}Programmier{-}Tutorials, die von 10 Doktoranden und Pos{-}Docs, einschließlich mir, gehalten wurden. Neben der Rekrutierung dieser Sprecher organisierte ich den Raum und die Ausrüstung für diese Sitzungen, warb für die Veranstaltungen und leitete die Sitzungen. Diese Serie war erfolgreich; Es wurde regelmäßig von 30{-}70 Forschern besucht.}%
\NewEntry{July 2016 and July 2017}{Trainer}{Einführung in die Programmierung in Python}{Dieser 4{-}tägige Workshop ist eine intensive Version des Semester{-}Python{-}Kurses, den ich an der LMU unterrichte. In diesem Zeitraum erlangen Studierende ohne Programmierkenntnisse die Fähigkeiten, die für die Datenanalyse und in Python erforderlich sind, und erklären ihren Analyse{-}Workflow.}%
\NewEntry{Summer 2016 and Summer 2017}{Data Science Dozent}{Einführung in die Programmierung in Python}{In diesem Semesterkurs, der zwei Jahre in Folge unterrichtet wurde, unterrichtete ich Programmieringenieure Datenmanagement, wissenschaftliche Datenanalyse und Programmierkenntnisse in einer neuen Sprache (Python). Neben der Organisation und Planung des Kurses habe ich auch alle Kursmaterialien, Hausaufgaben vorbereitet und ihre Abschlussarbeiten bewertet.}%
\NewEntry{Winter 2015}{Trainer}{Einführung in Matlab}{Ich plante und lehrte Matlab, mit der Programmierung von Studenten zu beginnen.}%
\NewEntry{December 2015}{Lehrassistent}{Psychophysics}{In diesem 2{-}wöchigen Blockkurs unterstützte ich Studenten bei der Programmierung und Analyse ihrer eigenen psychopysischen Experimente in Matlab, R und Excel.}%
\SubHeading{Forschungserfahrung}%
\NewEntry{May 2013 {-}\newline%
  Present}{Ludwig{-}Maximillians Universität}{Prof. Dr. Anton Sirota}{Programmierte eine 3D{-}Grafik{-}Engine in Python, um ein Virtual{-}Reality{-}System für frei bewegliche Ratten aufzubauen, konzipierte und realisierte kognitionswissenschaftliche Experimente und testete die Verallgemeinerbarkeit der Virtual{-}Reality{-}Forschung an ihren realen Gegenstücken; betreute sechs Studenten in Programmier{-}, Ingenieur{-} und kognitionswissenschaftlichen Projekten organisierte wöchentliche Journalclubs, organisierte geplante soziale Veranstaltungen und Retreats und bestellte neue Laborgeräte.}%
\SubHeading{Berufserfahrung}%
\NewEntry{}{Freiberuflich}{Wissenschaftlicher Berater und Trainer}{Trainiere ich Forscher in Programmierung, Experimentierdesign und wissenschaftlichen Schreibfähigkeiten, indem Sie sie dabei unterstützen, ihre eigenen Lösungen für Forschungsprobleme zu entwickeln und einwöchige Schulungen für ihre Forschungsinstitute durchzuführen.}%
\NewEntry{}{Forschungspraktikum}{The Neuromarketing Labs}{Ich habe die Einrichtung eines EEG{-}Labors abgeschlossen, einschließlich Software{-}Kalibrierung und Rauschmessungen. Entwarf und führte zwei Experimente durch, die die evozierten Reaktionen der semantischen Übereinstimmung und der Preisvereinbarung schätzten und analysierte dann die Daten. Die Ergebnisse des zweiten Experiments sind die Grundlage des kürzlich erschienenen Buches von Dr. Müller, Neuropricing. Derzeit ehrenamtliche Tätigkeit als EEG{-}Berater durch eintägige Workshops zu Fieldtrip, SPM und Artefaktkorrekturmethoden.}%
\SubHeading{wissenschaftliche Publikationen}%
\Description{Nicholas A. Del Grosso, Justin J. Graboski, Weiwei Chen, Eduardo Blanco Hernández, Anton Sirota. ``Virtual Reality system for freely{-}moving rodents.'' bioRxiv 161232. July 2017; doi=https://doi.org/10.1101/161232}%
\Description{Broetz D., Del Grosso, N.A., Rea M., Ramos{-}Murguialday, A., Soekadar S.R., Birbaumer, N. ``A New Hand Assessment Instrument for Severely Affected Stroke Patients.''  Journal of Neurorehabilitation. 2014; 34(3), 409{-}27.}%
\Description{Benoit, J.B., Del Grosso, N.A., Yoder, J.A., Denlinger, D.L. ``Resistance to Dehydration between Bouts of Blood Feeding in the Bed Bug, Cimex Lectularius, is Enhanced by Water Conservation, Aggregation, and Quiescence.'' American Journal of Tropical Medical Hygience. May 2007; 76(5), 987{-}93.}%
\end{cv}%
\end{document}